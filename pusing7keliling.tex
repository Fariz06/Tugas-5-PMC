\documentclass[conference]{IEEEtran}
\IEEEoverridecommandlockouts
% The preceding line is only needed to identify funding in the first footnote. If that is unneeded, please comment it out.
\usepackage{cite}
\usepackage{amsmath,amssymb,amsfonts}
\usepackage{algorithmic}
\usepackage{graphicx}
\usepackage{textcomp}
\usepackage{xcolor}
\def\BibTeX{{\rm B\kern-.05em{\sc i\kern-.025em b}\kern-.08em
    T\kern-.1667em\lower.7ex\hbox{E}\kern-.125emX}}
\begin{document}

\title{Perkalian Matriks\\
{\footnotesize \textsuperscript{*}Note: Sub-titles are not captured in Xplore and
should not be used}
\thanks{Identify applicable funding agency here. If none, delete this.}
}

\author{\IEEEauthorblockN{1\textsuperscript{st} Surya Dharma}
\IEEEauthorblockA{\textit{School of Electrical Engineering and Informatics)} \\
\textit{Institut Teknologi Bandung)}\\
Bandung, Indonesia\\
Email: 13220027@std.stei.itb.ac.id}

\and
\author{\IEEEauthorblockN{1\textsuperscript{st} Fariz Iftikhar Falakh}
\IEEEauthorblockA{\textit{School of Electrical Engineering and Informatics)} \\
\textit{Institut Teknologi Bandung)}\\
Bandung, Indonesia\\
Email: 13220029@std.stei.itb.ac.id}

\and
\author{\IEEEauthorblockN{1\textsuperscript{st}Senggani Fatah Sedayu}
\IEEEauthorblockA{\textit{School of Electrical Engineering and Informatics)} \\
\textit{Institut Teknologi Bandung)}\\
Bandung, Indonesia\\
Email: 13220035@std.stei.itb.ac.id}
}

\maketitle

\begin{abstract}
Perkalian matriks merupakan suatu operasi biner dari dua buah matriks yang menghasilkan sebuah matriks. Operasi biner merupapakan suatu operasi matematis, seperti penambahan, pengurangan, serta perkalian, yang menggabungkan dua elemen menjadi elemen lain.
\end{abstract}

\begin{IEEEkeywords}
Matrix, Strassen, Cannon, Naive
\end{IEEEkeywords}

\section{Introduction}
1. ada beberapa macam metode perkalian matriks
Perkalian matriks merupakan suatu operasi yang memiliki peran sentral dalam dunia matematis, serta memiliki jangkauan aplikasi yang sangat luas.  Karena banyaknya kegunaan operasi ini, banyak metode yang dikembangkan agar perkalian matriks, terutama untuk matriks dengan ukuran besar, menjadi lebih efisien. Pada laporan ini akan dibahas metode-metode tersebut serta pengaplikasiannya pada bahasa C.

\section{Studi Pustaka}

\subsection{Naive Algorithm}

ini penjelasan naive

\section{Strassen Algorithm}
ini penjelasan strassen

\subsection{Cannon Algorithm}\label{AA}
ini penjelasan cannon

\subsection{Fungsi Waktu}
ini penjelasan big o notation ama T(n)

\section{Metodologi Penelitian}
ini metodologi



\section{Implementasi dan Pengujian}

\subsection{Implementasi Naive Algorithm pada Bahasa C}

\subsection{Implementasi Strassen Algorithm pada Bahasa C}

\subsection{Implementasi Cannon Algorithm pada Bahasa C}


\section{Kesimpulan}
Kesimpulannya pusing tujuh keliling

\section*{References}



\begin{thebibliography}{00}
\bibitem{b1} G. Eason, B. Noble, and I. N. Sneddon, ``On certain integrals of Lipschitz-Hankel type involving products of Bessel functions,'' Phil. Trans. Roy. Soc. London, vol. A247, pp. 529--551, April 1955.
\bibitem{b2} J. Clerk Maxwell, A Treatise on Electricity and Magnetism, 3rd ed., vol. 2. Oxford: Clarendon, 1892, pp.68--73.
\bibitem{b3} I. S. Jacobs and C. P. Bean, ``Fine particles, thin films and exchange anisotropy,'' in Magnetism, vol. III, G. T. Rado and H. Suhl, Eds. New York: Academic, 1963, pp. 271--350.
\bibitem{b4} K. Elissa, ``Title of paper if known,'' unpublished.
\bibitem{b5} R. Nicole, ``Title of paper with only first word capitalized,'' J. Name Stand. Abbrev., in press.
\bibitem{b6} Y. Yorozu, M. Hirano, K. Oka, and Y. Tagawa, ``Electron spectroscopy studies on magneto-optical media and plastic substrate interface,'' IEEE Transl. J. Magn. Japan, vol. 2, pp. 740--741, August 1987 [Digests 9th Annual Conf. Magnetics Japan, p. 301, 1982].
\bibitem{b7} M. Young, The Technical Writer's Handbook. Mill Valley, CA: University Science, 1989.
\end{thebibliography}
\vspace{12pt}
\color{red}
IEEE conference templates contain guidance text for composing and formatting conference papers. Please ensure that all template text is removed from your conference paper prior to submission to the conference. Failure to remove the template text from your paper may result in your paper not being published.

\end{document}
