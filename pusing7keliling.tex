\documentclass[conference]{IEEEtran}
\IEEEoverridecommandlockouts
\usepackage{cite}
\usepackage{amsmath,amssymb,amsfonts}
\usepackage{algorithmic}
\usepackage[ruled]{algorithm2e}
\usepackage{graphicx}
\usepackage{textcomp}
\usepackage{xcolor}
\usepackage[colorlinks, urlcolor=blue]{hyperref}
\def\BibTeX{{\rm B\kern-.05em{\sc i\kern-.025em b}\kern-.08em
    T\kern-.1667em\lower.7ex\hbox{E}\kern-.125emX}}
\begin{document}


\title{Perkalian Matriks dengan Beberapa Algoritma}

\author{\IEEEauthorblockN{Surya Dharma}
\IEEEauthorblockA{\textit{School of Electrical Engineering and Informatics)} \\
\textit{Institut Teknologi Bandung)}\\
Bandung, Indonesia\\
Email: 13220027@std.stei.itb.ac.id}
\and
\IEEEauthorblockN{Fariz Iftikhar Falakh}
\IEEEauthorblockA{\textit{School of Electrical Engineering and Informatics)} \\
\textit{Institut Teknologi Bandung)}\\
Bandung, Indonesia\\
Email: 13220029@std.stei.itb.ac.id}
\and
\IEEEauthorblockN{Senggani Fatah Sedayu}
\IEEEauthorblockA{\textit{School of Electrical Engineering and Informatics)} \\
\textit{Institut Teknologi Bandung)}\\
Bandung, Indonesia\\
Email: 13220035@std.stei.itb.ac.id}
}

\maketitle

\begin{abstract}
Metode perkalian matriks yang umumnya diketahui adalah dengan perkalian baris kolom (dalam pemrograman perkalian baris kolom disebut \textit{Naive algorithm}).
Dengan berkembangnya ilmu matematika dan komputasi, diperoleh beberapa algoritma perkalian matriks yang memakan lebih sedikit waktu dibandingkan \textit{Naive algorithm}).
Penulis akan menggunakan beberapa algoritma untuk menghitung hasil perkalian 2 buah matriks dengan ukuran besar.
Lalu penulis akan membandingkan kompleksitas waktu dari algoritma-algoritma tersebut.
\end{abstract}

\begin{IEEEkeywords}
matriks, \textit{Naive algorithm}, algoritma Strassen, algoritma Cannon
\end{IEEEkeywords}

\section{Pendahuluan}
Perkalian matriks merupakan suatu operasi biner dengan operan dua buah matriks yang menghasilkan sebuah matriks.
Perkalian matriks memiliki peran penting dalam dunia matematika serta memiliki jangkauan aplikasi yang sangat luas, misalnya \textit{signal processing}).
Karena banyaknya kegunaan operasi ini, matematikawan mencoba untuk mencari metode perkalian matriks yang lebih memakan sedikit waktu.
Dari usaha tersebut, banyak metode yang dikembangkan agar perkalian matriks---terutama untuk matriks dengan ukuran besar---menjadi lebih efisien.

Algoritma-algoritma yang digunakan oleh penulis adalah \textit{Naive algorithm}, algoritma Strassen, dan algoritma Cannon.
Ketiga algoritma tersebut akan diimplementasikan dalam bahasa pemrograman C.
Penulis akan membandingkan seberapa cepat suatu algoritma menyelesaikan hasil perkalian 2 buah matriks dengan algoritma lain.

\section{Studi Pustaka}
\subsection{Naive Algorithm}

ini penjelasan naive

\subsection{Strassen Algorithm}
ini penjelasan strassen

\subsection{Cannon Algorithm}
Dalam \textit{computer science}, algoritma Cannon adalah algoritma terdistribusi (dirancang untuk dilakukan pada \textit{hardware} komputer yang terdiri dari berbagai prosesor)
untuk perkalian matriks untuk mesh (jaringan yang terbentuk dari sel dan titik) 2 dimensi.
Algoritma ini cocok digunakan untuk mesh N*N (atau matriks N*N) dan tidak cocok digunakan untuk matriks bukan persegi.

\begin{algorithm}
    \caption{Algoritma Cannon}
    \KwResult{Hasil perkalian matriks A berukuran n*n dan matriks B berukuran n*n}
    \KwSty{Procedure}\\
    \For{$i = 0$ \KwTo $n - 1$}{
        Geser sirkular kiri $baris[i]$ A $i$ kali
    }
    \For{$i = 0$ \KwTo $n - 1$}{
        Geser sirkular atas $kolom[i]$ B $i$ kali
    }
    \For{$k = 0$ \KwTo $n - 1$}{
        \For{$i = 0$ \KwTo $n - 1$}{
            \For{$j = 0$ \KwTo $n - 1$}{
                $C[i][j] \leftarrow C[i][j] + A[i][j] \times B[i][j]$
            }
        }
        Geser sirkular kiri semua baris A 1 kali\\
        Geser sirkular atas semua baris B 1 kali
    }
\end{algorithm}

\subsection{Fungsi Waktu}
ini penjelasan big o notation ama T(n)

\section{Metodologi Penelitian}
ini metodologi

\section{Implementasi dan Pengujian}

\subsection{Implementasi Naive Algorithm pada Bahasa C}

\subsection{Implementasi Strassen Algorithm pada Bahasa C}

\subsection{Implementasi Cannon Algorithm pada Bahasa C}


\section{Kesimpulan}
Kesimpulannya pusing tujuh keliling

\begin{thebibliography}{00}
\bibitem{b1} G. Eason, B. Noble, and I. N. Sneddon, ``On certain integrals of Lipschitz-Hankel type involving products of Bessel functions,'' Phil. Trans. Roy. Soc. London, vol. A247, pp. 529--551, April 1955.
\bibitem{b2} J. Clerk Maxwell, A Treatise on Electricity and Magnetism, 3rd ed., vol. 2. Oxford: Clarendon, 1892, pp.68--73.
\bibitem{b3} I. S. Jacobs and C. P. Bean, ``Fine particles, thin films and exchange anisotropy,'' in Magnetism, vol. III, G. T. Rado and H. Suhl, Eds. New York: Academic, 1963, pp. 271--350.
\bibitem{b4} K. Elissa, ``Title of paper if known,'' unpublished.
\bibitem{b5} R. Nicole, ``Title of paper with only first word capitalized,'' J. Name Stand. Abbrev., in press.
\bibitem{b6} Y. Yorozu, M. Hirano, K. Oka, and Y. Tagawa, ``Electron spectroscopy studies on magneto-optical media and plastic substrate interface,'' IEEE Transl. J. Magn. Japan, vol. 2, pp. 740--741, August 1987 [Digests 9th Annual Conf. Magnetics Japan, p. 301, 1982].
\bibitem{b7} M. Young, The Technical Writer's Handbook. Mill Valley, CA: University Science, 1989.
\end{thebibliography}

\end{document}
